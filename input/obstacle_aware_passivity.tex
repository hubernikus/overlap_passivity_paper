\section{Obstacle Aware Passivity} \label{sec:obstacle_aware_passivity}
We propose a novel controller, which ensures passivity as defined in \eqref{eq:control_command} but adapts the damping matrix given in \eqref{eq:damping_matrix} based on the desired velocity $\dot{\vecs \xi}$ and obstacles in the surrounding. 
Hence, the damping matrix smoothly changes from being aligned with the direction of the velocity, as used in \cite{kronander2015passive}, to be aligned with the averaged normal of the obstacles.
Far away the system is designed to follow the initial velocity, but when approaching the obstacle, it increases the damping, which decreases the chance of a collision.

The desired damping matrix transitions smoothly between two states: velocity preserving, and collision avoidance. It is defined as a linear combination:
\begin{equation}
    \matd D(\vecs\xi) = \left(1 - w(\vecs\xi) \right) {\matd D^{f}}(\vecs\xi) + w(\vecs\xi)  {\matd D^{\mathrm{obs}}}(\vecs\xi) \label{eq:damping_summation}
\end{equation}

The danger $w(\vecs\xi) \in [0, 1]$ indicates the proximity to the obstacles based on the distance function $\Gamma_o(\vecs \xi)$. Far away from obstacles the weight reaches $w(\vecs \xi) = 0$, whereas $w(\vecs \xi) = 1$ when approaching a boundary:
\begin{equation}
  \begin{split}
w(\vecs\xi) =
\max \left(0,  \frac{\Gamma^{\mathrm{crit}} - \Gamma(\vecs\xi)}{\Gamma^{\mathrm{crit}} - 1} \right) \| \vecs n(\vecs \xi) \| \\
\text{with} \quad
\Gamma(\vecs\xi) = \min_{o = 1..N^{\mathrm{obs}}} \Gamma_o(\vecs\xi)
\label{eq:weight_function}
\end{split}
\end{equation}
The critical distance $\Gamma^{\mathrm{crit}} \in \mathbb{R}$ defines the distance where the system has higher damping towards the obstacle.

Note that, since ${\matd D^f}(\vecs\xi)$ and $\matd {D^{\mathrm{obs}}}(\vecs\xi)$ follow design given in \eqref{eq:damping_matrix} and are positive semi-definite matrices, thus $\matd {D}(\vecs\xi)$ is positive semi-definite, too.

\begin{figure}
  \center
  \includesvg[width=0.7\columnwidth]{figures/damping_basis_construction.svg}
\caption{The damping matrix enforcing desired velocity following $\matd{D}^{f}$ has the first basis vector $\vect q_1^{f}$ which follows the avoidance velocity $\vect f(\vecs \xi)$ and damping to enforce collision avoidance with $\matd{D}^{\mathrm{obs}}$ uses the normal $\vect n(\vecs \xi)$ to construct the first direction of the decomposition basis $\vect q^{\mathrm{obs}}_1$.}
\label{fig:damping_basis_construction}
%% \label{fig_basis_3D_DS_obs}
\end{figure}

\subsection{Collision Repulsion Damping} \label{sec:obstacle_repulsion}

\subsubsection{Normal Direction}
The damping matrix $\matd D^{\mathrm{obs}}(\vecs \xi)$ rejects velocities in the direction of the obstacles. To allow this, we introduce an average normal direction:
\begin{equation}
  \vecs n(\vecs\xi) = \sum_{o=1}^{N^{\mathrm{obs}}} \vecs{n_o}(\vecs\xi)
  \frac{1 / (\Gamma_o(\vecs \xi) - 1)}{\sum_{p=1}^{N^\mathrm{obs}} 1 / (\Gamma_p(\vecs \xi) - 1)}
  \label{eq:averaged_normal}
\end{equation}
 where the unit normals $\vecs{n_o} (\vecs\xi)$  are pointing away from the obstacle $o = 1,  ..,  N^{\mathrm{obs}}$, see Figure~\ref{fig:resultant_normal}. 

The averaged normal $\vecs n(\vecs \xi)$ is a weighted linear combination of the obstacles' normals, giving more importance to closer obstacles.
Additionally, the averaged normal converges to an obstacle normal as we converge towards it, i.e., $\lim_{\Gamma_o(\vecs \xi) \rightarrow 1} \vecs n(\vecs \xi) = \vecs n_o(\vecs \xi)$.
Note that the averaged normal is a zero-vector when two obstacles oppose each other. This case will be further discussed in the next section.

\subsubsection{Decomposition Matrix}
The decomposition matrix $\matd Q^{\mathrm{obs}}(\vecs \xi)$ has its first vector aligned with the normal to the obstacle:  $\vecs q_1^{\mathrm{obs}}(\vecs\xi) =  \vecs n(\vecs\xi) / \lVert\vecs n(\vecs\xi)\rVert$ 

The second vector is set to be aligned with the desired velocity as much as possible, this will allow increased velocity following (Fig.~\ref{fig:damping_basis_construction}). However, it has to remain orthonormal to $\vect q_1^{\mathrm{obs}}$
\begin{equation}
  \vecs q_2^{\mathrm{obs}} = \frac{\hat{\vecs q}_2^{\mathrm{obs}}}{\| \hat{\vecs q}_2^{\mathrm{obs}} \|}
  \quad
  \hat{\vecs q}_2^{\mathrm{obs}} = \vecs q_1^{f} - \vecs q_1^{\mathrm{obs}} p \quad  \forall \, \vecs \xi : | p | < 1
\end{equation}
where the object weight is given as $p = \dotprod{\vecs q_1^{\mathrm{obs}}}{\vecs q_1^{f}}$
For the case that $| p | = 1$, the second basis $\vecs q_2^{\mathrm{obs}}$ is set to be any orthonormal vector. The remaining vectors $\vecs q_i^{\mathrm{obs}}, i = 3, .., N$ are constructed to form an orthonormal basis to the first two.

\subsubsection{Damping Values}
Analogously, we smoothly define the values of the diagonal matrix $\matd{S}^{\mathrm{obs}}(\vecs \xi)$:
\begin{equation}
  \matd{S}_d^{\mathrm{obs}}(\vecs \xi) =
  \begin{cases}
    s^{\mathrm{obs}} & d = 1 \\
    | p | s^c + (1 - | p |) s^{f} & d = 2 \\
    s^c & d \geq 3 
  \end{cases}
  \quad d = 1 .. N\
\end{equation}
where the damping along the nominal direction $s^{f} \in \mathbb{R}$, obstacle-damping $s^{\mathrm{obs}} \in \mathbb{R}$, and the compliant-damping $s^c \in \mathbb{R}$ are user-defined parameters which define the behavior of the passive-controller.

The first entry of $\matr{S}^{\mathrm{obs}}$ indicates the damping towards the obstacle, and the second entry indicates the desired velocity following. Note how the later value decrease, when normal and velocity are becoming parallel.

\subsubsection{Damping Only Towards Obstacle} \label{sec:damping_only_toward}
To reject only the disturbances that push the agent against the obstacle and allow compliance away from the obstacle, we have added a new feature. Hence, in case the robot is already moving away, i.e., $\vecs{\dot\xi}^T \vecs n(\vecs\xi) > 0$,  the value of $\matd{S}^{\mathrm{obs}}_1$ gets overwritten with $s^{c}$, the general, compliant damping value:
\begin{equation}
  \matd{S}_1^{\mathrm{obs}} =
  \begin{cases}
    s^{\mathrm{obs}} & \text{if} \;\; \left(\vect f(\vecs \xi) - \vecs{\dot \xi} \right)^T \vecs n(\vect \xi) > 0 \\
    s^{\mathrm{c}} & \text{otherwise}
  \end{cases}
  \label{eq:leaving_compliance}
\end{equation}
This ensures that the agent can move freely away from the obstacle, but is damped when pushed towards.


\subsection{Velocity Preserving Damping}
The decomposition matrix $\matd Q^{f}$ is an orthonormal basis with the first vector being parallel to the desired velocity $\vect f(\vecs\xi)$. Let us define $\vecs q^{f}_1 (\vecs\xi) = \vect f({\vecs \xi}) / \lVert \vect f({\vecs \xi}) \rVert$, as proposed in Section~\ref{sec:trad_passive}. Hence, the values of the diagonal matrix $\matd S^{f}$ are high in the direction of the desired velocity (first component), but more compliant in the remaining directions. 
However, we need to additionally ensure that when moving in a narrow passage between two obstacles, where the normal vector cancels $\vect n(\vecs \xi) \approx 0\vect 0$, but a low distance value $\Gamma(\vecs \xi) \approx 1$, the damping perpendicular to the velocity direction is high. Hence, we propose:
\begin{equation}
  \begin{split}
  \matd{S}^{f}_{d} =
  \begin{cases}
    s^{f} & d = 1 \\
    w^p s^{\mathrm{obs}} + (1- w^p) s^s & d \geq 2 
  \end{cases} \quad d = 1 .. N\\
  \text{with} \quad
  %w^p = \min \left(1,  \| \vecs n(\vecs \xi) \|^2 + \left(\frac{\Gamma(\vecs \xi) -1}{\Gamma^{\mathrm{crit}} - 1}\right) ^2 \right)
   w^p = \min \left(1,  \| \vecs n(\vecs \xi) \|^2 + \left(\frac{\Gamma^{\mathrm{crit}} -\Gamma(\vecs \xi)}{\Gamma^{\mathrm{crit}} - 1}\right) ^2 \right)
  \end{split}
\end{equation}

\subsection{Damping Parameter Design}
In general, higher values result in improved velocity following and disturbance repulsion, whereas lower values allow more compliant behavior.

The damping value in the direction of the obstacle is set high $s^{\mathrm{obs}}$ to ensure obstacle avoidance with any possible disturbance. 
Conversely, the damping in the direction of the velocity $s^{f}$ is set medium to high, as the system should follow the desired velocity $\vect f(\vecs \xi)$, however, it should still remain compliant if needed.
Finally, for all other directions, a low damping value $s^{c}$ should be chosen to facilitate interaction.
This can be summarized as:
\begin{equation}
s^{\mathrm{obs}} > s^{f} \gg s^{c} > 0
\end{equation}
