\subsection{Problem Statement}
Following assumptions are made about the desired velocity $\vecs f(\vecs\xi)$:
\begin{enumerate}
    \item $\vecs f(\vecs\xi)$ is continuous for all reachable states.
    %% \item $\vecs f(\vecs\xi)$ has a single equilibrium point $\vecs\xi^a$, such that $\{ \vecs \xi : \vecs f(\vecs \xi) = \vecs 0 \} = \{ \vecs\xi^a \}$. 
    \item $\vecs f(\vecs\xi)$ is bounded, i.e., there exists a constant $v^{\mathrm{max}} \in \mathbb{R}$ such that $\| \vecs f(\vecs\xi) \| \leq v^{\mathrm{max}} \;\; \forall \, \vecs \xi \in \mathbb{R}^N$
    \item $\vecs f(\vecs\xi)$ leads to a collision-free motion, i.e., $\vecs{n_o}(\vecs\xi)^T \vecs f(\vecs\xi) \geq 0$ as $\Gamma_o(\vecs \xi) \rightarrow 1 \quad \forall o = 1 .. N^{\mathrm{obs}}$ with the normal $\vecs n_o$ and distance $\Gamma_o$ of the $o$-th obstacle. 
\end{enumerate}

Note that velocity obtained using the obstacle avoidance method described in \eqref{eq:modulated}, fulfills these conditions if base velocity $\vecs f^b(\vecs \xi)$ is continuous and bounded.
