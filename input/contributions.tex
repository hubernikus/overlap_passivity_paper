\subsection{Contribution}
We introduce a passive controller that incorporates into the feedback control loop as visualized in Figure~\ref{fig:control_scheme_passive}. 
In this work, we make the following contributions:
\begin{itemize}
\item The design of the obstacle-aware passive controller
(Section~\ref{sec:obstacle_aware_passivity})
\item A passivity guarantee (without the need for a storage tank) which applies to general damping controllers (Theorem~\ref{theorem:passivity})
\item A collision avoidance analysis which provides insight into the path consistency around obstacles (Section~\ref{sec:collision_avoidance})
\item Discrete-time analysis to enable control parameter design which ensures stability (Section~\ref{sec:discrete_time_behavior})
\item Implementation and testing on 7DoF robot arm (Section~\ref{sec:evaluation})
\end{itemize}

\ifthesis
\,
\else
\begin{figure}[thb]
  \center
  \includesvg[width=1.0\columnwidth]{figures/control_scheme_passive.svg}
\caption{The desired velocity $\vect f^b(\vecs \xi)$ can result from a learned velocity field or pointing towards a desired attractor $\vecs \xi^a$. The desired velocity is used to evaluate the obstacle avoidance velocity $\vect f(\vecs \xi)$, which is fed into the force controller to obtain the control force $\vect \tau_c$. In order to achieve collision avoidance, the distance function $\Gamma_o(\vecs \xi)$, the normal direction $\vect n_o(\vecs \xi)$, and the reference direction $\vect r_o(\vecs \xi)$ are evaluated for each obstacle $o = 1 .. N^\mathrm{{obs}}$.}
\label{fig:control_scheme_passive}
\end{figure}
\fi
